\documentclass[DM,lsstdraft,authoryear,toc]{lsstdoc}
% lsstdoc documentation: https://lsst-texmf.lsst.io/lsstdoc.html

% Package imports go here.
\usepackage{graphicx}
\usepackage{url}
\usepackage{latexsym}
\usepackage{color}
\usepackage{enumitem}

% Local commands go here.

% To add a short-form title:
% \title[Short title]{Title}
\title[Host Association]{Host Galaxy Association for {\tt DIAObjects}}

% Optional subtitle
% \setDocSubtitle{A subtitle}

\author{%
M.~L.~Graham et al.
}

\setDocRef{DMTN-TBD}
\date{\today}
% Optional: name of the document's curator
% \setDocCurator{The Curator of this Document}
\setDocUpstreamLocation{\url{https://github.com/lsst-dm/dmtn-xxx}}

\setDocAbstract{%
This document argues that, in order to better enable extragalactic transient science with brokers, two new {\tt DIAObject} catalog elements should be computed and included in the alert packets: (1) the {\tt objectId} for the three {\tt Object} catalog galaxies with the lowest separation distance (based on the galaxy's 2D luminosity profile) from the {\tt DIAObject}, and (2) the separation distances for those three {\tt Objects}.
}

% Change history defined here.
% Order: oldest first.
% Fields: VERSION, DATE, DESCRIPTION, OWNER NAME.
% See LPM-51 for version number policy.
\setDocChangeRecord{%
  \addtohist{0}{2020-02-26}{Inception.}{Melissa Graham}
}

\begin{document}

% Cite requirements using these macros.
% \lsrreq \ossreq \dmreq \reqparam 

% Create the title page.
% Table of contents is added automatically with the "toc" class option.
\maketitle

% % % % % % % % % % % % % % % % % % % % % % % % % %
\section{Introduction} \label{sec:intro}

LSST will issue alert packets within $60$ seconds for all sources detected during difference image analysis (DIA; {\tt DIASources}), which are associated by sky coordinate into objects ({\tt DIAObject}). 
Individuals and brokers processing alerts will use the information in these packets to rapidly evaluate and prioritize {\tt DIAObjects} for follow-up with limited resources.
Thus the contents of the alert packet have been designed to contain a sufficient amount of LSST data about each {\tt DIAObject} to enable immediate analysis.

One important piece of information is the association of each {\tt DIAObject} with a static-sky {\tt Object} from the Data Release catalogs.
Brokers will use the {\tt Object} association to obtain data about the static-sky object from the DR catalogs, such as whether it might be galactic or extragalactic, at high- or low-redshift, nuclear or offset from a host, etc.
All of this information can help an alert stream user identify and prioritize their targets of interest, and delivering alerts \emph{with} the static-sky association already completed avoids the situation of multiple users cross-matching in real time.
The {\tt Object} association for {\tt DIAObjects} will of course also be used by scientists working with the Prompt or Data Release {\tt DIAObjects} catalogs on longer timescales, not just alert consumers.
However, the main goal of this document is to assess the best option -- from a scientific perspective -- for {\tt Object} association that can be completed during the 60 second Alert Production timescale.

Below, Section \ref{sec:DMplans} describes the current plan for associating {\tt DIAObjects} with the DR {\tt Objects} catalog; Section \ref{sec:options} presents and discusses options that have been used successfully by other surveys; and Section \ref{sec:recs} makes recommendations for additions to the {\tt DIAObject} catalog to improve associations for extragalactic transients and their host galaxies.


\section{Current DM Plans for Host Galaxy Association}\label{sec:DMplans}

With respect to associations between Prompt {\tt DIAObjects} and Data Release {\tt Objects}, the contents of the alert packet as defined in \citeds{LSE-163} includes the following:
\begin{itemize}
\item {\tt nearbyObj} ({\tt unit64[6]}), the {\it "closest {\tt Objects} (3 stars and 3 galaxies) in Data Release database"}
\item {\tt nearbyObjDist} ({\tt float[6]}), the {\it "distances to {\tt nearbyObj}"} in arcseconds
\item {\tt nearbyObjLnP} ({\tt float[6]}), the {\it "natural log of the probability that the observed {\tt DIAObject} is the same as the nearby {\tt Object}"}
\end{itemize}
For the latter, there is a footnote that says {\it "This quantity will be computed by marginalizing over the product of position and proper motion error ellipses of the {\tt Object} and {\tt DIAObject}, assuming an appropriate prior"}.

The current definitions of {\tt nearbyObj}, {\tt nearbyObjDist}, and {\tt nearbyObjLnP} are not as useful as they could be for transients in host galaxies. 
For extragalactic transients, the three nearest galaxies are not always the three most likely host galaxies, and the distance in arcseconds matters less than a separation distance that accounts for the galaxies' spatial luminosity profiles.
Furthermore, the definition of {\tt nearbyObjLnP} is only appropriate for static variable point sources (stars): for transients in host galaxies, the observed {\tt DIAObject} will never be ``the same as the nearby {\tt Object}".


\section{Options to Improve Host Galaxy Association}\label{sec:options}

Statistically, the most likely host for a given transient is the galaxy which contributes the most optical flux at the transient's location.
This is usually estimated by calculating a \emph{separation distance} from the nearby galaxies to the transient which is expressed in terms of the galaxy's spatial luminosity profile, and then assuming the galaxy with the lowest separation distance is the host.
The following are several options for estimating which nearby galaxy is the most likely host of an extragalactic transient.

\subsection{Effective Radius}

For the separation distance, use the radial distance from the core of the galaxy to the location of the transient, divided by the effective radius of the galaxy.
The DR {\tt Objects} table is already planned to contain suitable effective radii such as the parameter {\tt kronRad90} \citedsp{LSE-163}.
This option for the separation distance would not require any additional processing aside from dividing the radial distance from transient to {\tt Object} by the effective radius of the {\tt Object}.
Although this kind of separation distance would account for the relative sizes of the potential host galaxies, it does not account for their position angles, and so would not be as accurate for assessing potential high-inclination host galaxies.

\subsection{Second Moments}

Calculate a separation distance based on the two-dimensional luminosity profile of the nearby galaxies.
For example, \citet{2006ApJ...648..868S} describe the method applied to the Supernova Legacy Survey (SNLS), using a separation distance of $R^2 = C_{xx} x_r^2 + C_{yy} y_r^2 + C_{xy} x_r y_r$, where $C_{xx}$, $C_{yy}$, and $C_{xy}$ are ellipse parameters derived from the second moments of the galaxy luminosity profile and $x_r$,$y_r$ are the on-sky distances between the centroids of the transient and the galaxy.
The DR {\tt Objects} table is already planned to contain the second moments of the galaxy luminosity profiles ({\tt Ixx}, {\tt Iyy}, and {\tt Ixy}; \citeds{LSE-163}).
A step-by-step description of how this separation distance can be calculated from planned {\tt DIAObject} and {\tt Object} table elements is provided in Section \ref{sec:app}.

Multiple recent surveys have used this method (or similar) to associate transients with their host galaxies, such as \citet{2018PASP..130f4002S} for SDSS supernovae, and \citet{2016AJ....152..154G} who use real and simulated data to evaluate the optimal method for host association.
This option for the separation distances requires slightly more computational steps, but would account for both the relative sizes and position angles of the potential host galaxies. 


\subsection{2D Algorithms}

Aside from adopting a separation distance, there are more complicated methods for identifying the most likely host for a given transient.
For example, the nearby galaxy with the smallest fraction of light interior to an isophot through the transient's location, where the isophot shape is given more degrees of freedom and not constrained to concentric ellipticals as in the second moment method above.
Another example is to use an algorithm that provides deblended footprints for nearby extended objects, and that can estimate the fraction of light in given pixel that should be attributed to each (e.g., as the SCARLET deblender can do, \cite{2018A&C....24..129M}).
The most likely host galaxy would be the one which contributes the most flux at the pixel location of a transient.
While all DR {\tt Objects} will be associated with a footprint (a region of connected pixels), the footprint information will not be stored in the {\tt Object} table.
The use of footprints in identifying potential host galaxies would require more computational resources during Prompt processing, but would probably return a more accurate host association for only a very small fraction of {\tt DIAObjects}. 

\subsection{Hostless Transients}

For some scientific analyses, transients which are $>$3-5 effective radii away from the nearest galaxy, or for which $>99\%$ of the potential host's luminosity is within the separation distance, are considered ``hostless" (e.g., \citealt{2011ApJ...729..142S}).
Such a cutoff has been appropriate for past samples of $\sim$hundreds of transients, but will not be appropriate for the LSST sample size.
Furthermore, the decision of whether and how to consider a transient ``hostless" is best left as a scientific decision for the end-user.
Thus, no such cut should be applied during the association of {\tt DIAObjects} and {\tt Objects}, and the most likely hosts should still be reported, even if the probability is low.

\subsection{Galaxy/Transient Types}

The association of transients with their host galaxy can be more accurate if their properties are also considered.
For example, the potential host galaxy's redshifts can be used to calculate separation distances in physical units, or to estimate the absolute brightness of the transient and consider whether it is physically plausible.
Priors based on the established correlations between transient types and host galaxy morphology or color can also be used to refine a probabilistic host association, such as how core collapse supernovae are almost always associated with star formation (except for a few notable cases, e.g., \citealt{2012ApJ...753...68G,2019ApJ...887..127I}).
A demonstration that these correlations between host and transient types are so robust that the host type can be used to provide a statistical classification of the transient type was presented by \citet{2013ApJ...778..167F}.

% At DESC, Alex Gagliano presented their work on host association using PCA on teh galaxy properties, which shows that color and morphology are the first two PCA components, and that the hosts of SNIa/SNII etc. occupy different parts of the PC1 vs PC2 plane. That paper is still in prep. but could maybe be cited here eventually.

However, making science-informed associations between transients and galaxies based on any kind of derived properties is beyond the scope of Prompt processing, and is best left to the users on a case-by-case basis. 
Thus, properties of the transients and/or the nearby galaxies (beyond their coordinates and luminosity profile) should not be used during the association of {\tt DIAObjects} and {\tt Objects}.
For a very thorough assessment of an optimized, science-driven system for associating supernovae and their host galaxies -- including the role and performance of machine learning methods -- we direct the reader to \citet{2016AJ....152..154G}.

\section{Recommendations}\label{sec:recs}

For the ten\footnote{This is a conservative estimate.} {\tt Object} catalog galaxies (stars excluded) that are nearest to a given {\tt DIAObject} in terms of radial distance, a separation distance should be calculated with respect to the transient location using the second moments of each galaxy's luminosity profile (as described in Section \ref{sec:app}).

Two new {\tt DIAObject} catalog elements should be added: {\tt nearbyPotHost}, containing the objectId for the three galaxies with the lowest separation distances, and {\tt nearbyPotHostSepDist}, containing the separation distances for those three galaxies.

An analog for the existing element {\tt nearbyObjLnP}, which represents the probability of association for static but variable point sources (stars), is not necessary for potential host galaxies. The existing {\tt DIAObject} catalog elements {\tt nearbyObj} and {\tt nearbyObjDist} can remain unchanged.

This would add unit64[3] and float[3] to the {\tt DIAObject} catalog and to each alert ? a small and worthwhile addition.


\subsection{Draft RFC}\label{ssec:draft_rfc}

\textit{The following text should be posted as a Request For Comments (RFC) in Jira, and at the same time this DMTN should be made official and available.}

In order to better enable extragalactic transient science with brokers, it is proposed that two new {\tt DIAObject} table elements be computed during Alert Production:
(1) {\tt nearbyPotHost}, containing the {\tt objectId} for the three {\tt Object} catalog galaxies with the lowest \emph{separation distances}, and
(2) {\tt nearbyPotHostSepDist}, the separation distances for those three {\tt Objects}.
The \emph{separation distance} should be calculated with respect to the transient location using the second moments of each galaxy's luminosity profile, as described in detail in Section \ref{sec:app} of DMTN-XXX.
All parameters required for the calculation of the separation distances are already planned to be in the {\tt DIAObject} and {\tt Object} tables, and this change would only add unit64[3] and float[3] per {\tt DIAObject} catalog entry, and per alert.

The above changes would be incorporated into the Data Products Definitions Document.




\clearpage
\section{Appendix: Separation Distance from Second Moments}\label{sec:app}

From the LSST catalogs the following table elements are used to define the parameters needed to calculate the separation distance \citedsp{LSE-163}:
\begin{center}
\begin{tabular}{ccll}
\hline
Parameter & Unit & Table Element & Description \\
\hline
$x_{\rm trans},y_{\rm trans}$ & $\rm degrees$ & {\tt DIAObject} {\tt radec} & transient centroid  \\
$x_{\rm gal},y_{\rm gal}$       & $\rm degrees$ & {\tt Object} {\tt radec}       & galaxy centroid      \\
$\overline{x^2}$,$\overline{y^2}$,$\overline{xy}$ & $\rm arcsec^2$  & {\tt Object} {\tt Ixx}, {\tt Iyy}, {\tt Ixy} & galaxy second moments \\
\hline
\end{tabular}
\end{center}

As described in Section 10 of E. Bertin's Source Extractor manual\footnote{Version 2.3: \url{https://www.astromatic.net/pubsvn/software/sextractor/trunk/doc/sextractor.pdf}} (and presumably many other places), the unitless ellipse parameters $C_{xx},C_{yy},C_{xy}$ can be calculated from the second moments via:
\begin{equation}
C_{xx} = \frac{\overline{y^2}}{\sqrt{ \left( \frac{\overline{x^2}-\overline{y^2}}{2} \right)^2 + \overline{xy}^2}}
\end{equation}
\begin{equation}
C_{yy} = \frac{\overline{x^2}}{\sqrt{ \left( \frac{\overline{x^2}-\overline{y^2}}{2} \right)^2 + \overline{xy}^2}}
\end{equation}
\begin{equation}
C_{xy} = -2 \frac{\overline{xy}}{\sqrt{ \left( \frac{\overline{x^2}-\overline{y^2}}{2} \right)^2 + \overline{xy}^2}}
\end{equation}

The sky distances between the transient and galaxy centroids are calculated as follows, and include the cos-dec factor and a conversion from units of degrees to arcseconds:
\begin{equation}
x_r = 3600(x_{\rm SN} - x_{\rm gal})
\end{equation}
\begin{equation}
y_r = 3600(y_{\rm SN} - y_{\rm gal})\cos{y_{\rm gal}}
\end{equation}

Finally, the separation distance $R$ in arcseconds is calculated as:
\begin{equation}
R^2 = C_{xx} x_r^2 + C_{yy} y_r^2 + C_{xy} x_r y_r.
\end{equation}


\clearpage
% Include all the relevant bib files.
% https://lsst-texmf.lsst.io/lsstdoc.html#bibliographies
\bibliography{local,lsst,lsst-dm,refs_ads,refs,books}

\end{document}
